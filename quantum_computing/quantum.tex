\documentclass{beamer}
\usepackage{tabularx}
\usepackage{beamerthemeCambridgeUS}
%\usetheme{Antibes}
\usepackage{hyperref}
% for themes, etc.
%\mode<presentation>
%{ \usetheme{boxes} }

\usepackage{times}  % fonts
\usepackage{graphicx}



%colours
\definecolor{lightblue}{rgb}{0.1, 0.1, 0.6}
\definecolor{maroon}{rgb}{0.3, 0.1, 0.7}

\newcommand{\FR}[2]{
	{\textstyle \frac{#1}{#2} }}
\def\RR{\mathbb{R}}


% macros
\newcommand{\hhq}{{\scriptstyle{{\frac{1}{4}}}}}
\newcommand\hf{\textstyle{1\over 2 }\displaystyle}
\newcommand\hhf{\scriptstyle{1\over 2 }\displaystyle}
\newcommand{\erf}{\mathrm{erf}}
\def\h{\textcolor{red}{\mathbf{h}}}
\def\z{\textcolor{maroon}{\mathbf{z}}}
\newcommand{\zave}{\z_{\mathrm{ave}}}
\def\by{\textcolor{lightblue}{\mathbf{y}}}
\def\bv{\textcolor{blue}{\mathbf{v}}}
\def\bx{\textcolor{red}{\mathbf{x}}}
\def\bp{\textcolor{maroon}{\mathbf{p}}}
\makeatother
\setbeamertemplate{footline}
{
	\leavevmode%
	\hbox{%
		\begin{beamercolorbox}[wd=.4\paperwidth,ht=2.25ex,dp=1ex,center]{author in head/foot}%
			\usebeamerfont{author in head/foot}\insertshortauthor
		\end{beamercolorbox}%
		\begin{beamercolorbox}[wd=.6\paperwidth,ht=2.25ex,dp=1ex,center]{title in head/foot}%
			\usebeamerfont{title in head/foot}\insertshorttitle\hspace*{3em}
%			\insertframenumber{} / \inserttotalframenumber\hspace*{1ex}
		\end{beamercolorbox}}%
	\vskip0pt%
}
\makeatletter
\setbeamertemplate{navigation symbols}{}




% these will be used later in the title page
\title{Quantum Computing}
\author{Gahan M. Saraiya (18MCEC10)}
\institute{M.Tech (Computer Science and Engineering) 
	\\ Institute of Technology, Nirma University, Ahmedabad}
\date{{\scriptsize September 2018}}

% note: do NOT include a \maketitle line; also note that this title
% material goes BEFORE the \begin{document}

% Recurring Outline for every section, with highlighting
\AtBeginSection[]
{ \begin{frame}<beamer> 
	\frametitle{Outline of Talk}
	\tableofcontents[currentsection]%[pausesections]
\end{frame} }

\begin{document}

\begin{frame}
\titlepage
\end{frame}


\section{Introduction}

	\begin{frame}[allowframebreaks]
	\frametitle{Keywords}
		\begin{itemize}
			\item quantum
				{\scriptsize \\ In early 1900's quantum theory developed, successfully explaining weird behavior of atoms/electrons
				\\ In late $20^{th}$ century it was discovered that it can be applied to information itself}
			\item quantum computer
				{\scriptsize \\ A device capable of controlling quantum states in fashionable way (the way ordinary computer controls/manipulates bits)}
			\item qubit
				{\scriptsize \\ quantum version of bits (values: $|0>$, $|1>$, or both at once), superposition phenomenon
				($|>$ used to distinguish qubits from ordinary bits)}
			\item \href{https://en.wikipedia.org/wiki/Superposition_principle}{superposition}
				{\scriptsize \\ weighted sum or difference of two or more states}
			\item \href{https://en.wikipedia.org/wiki/Quantum_superposition}{quantum superpositions}
				{\scriptsize \\ predicts that a computer with $N$ qubits can exist in superposition of all $2^N$ (all at the same time), which is exponentially higher than ordinary/classical computer}
			\item {entanglement}
				{\scriptsize \\ A property of quantum superpositions}
		\end{itemize}
	%\vspace{1cm}
	\end{frame}
	
	
	\begin{frame}
	\frametitle{Why Quantum Computer?}
	
	\begin{itemize}
		\item Moore's Law
		\item components becoming smaller - atomic scale?
		\item quantum effects
			\\ {\scriptsize 
				it will start to play role even before we reach transistors that are only one atom large
			}
		\item build quantum computers that embrace quantum effects
			\\ {\scriptsize 
				making use of quantum physics to build different computers than traditional one
			}
	\end{itemize}
	\end{frame}


\section{Quantum Architecture}

\begin{frame}[allowframebreaks]
\frametitle{Applications}
	\begin{itemize}
		\item \textbf{Optimization}
			\begin{itemize}
				\item systems design
				\item airline scheduling
				\item mission planning
				\item financial analysis
				\item web search
				\item cancer radiotherapy
				\item Volkswagen was the first car manufacturer to use a quantum computer to calculate traffic flows. 
				\item Recruit Communications and D-Wave - collaborated to apply quantum computing to marketing, advertising and communications. (optimize the efficiency of matching advertisements to customers for web advertising.)
			\end{itemize}
		\newpage
		
		\item \textbf{Machine Learning}
		\begin{itemize}
			\item Improving forecast capability with neural network
			\item learn to recognize essences of objects by recognizing patterns in huge amount of data
			\item native capability of D-Wave Quantum Processing Unit (QPU)
			\item NASA scientists trained the D-Wave 2X system on image data sets in a generative unsupervised learning 
		\end{itemize}
		\newpage
		
		\item \textbf{Biomedical Simulations}
		\begin{itemize}
			\item simulate molecular structures
			\item Using D-wave One quantum computer researchers from Harvard University solved the puzzle of how some proteins fold in year 2012 
		\end{itemize}
		\item \textbf{Biomedical Simulations}

	\end{itemize}
\end{frame}

\begin{frame}[allowframebreaks]
\frametitle{Top Quantum Computing Companies}
	\begin{itemize}
		\item \href{https://www.dwavesys.com/}{D-wave Systems}
			\\ {\scriptsize
				World's first Quantum Computing Company
				\\ integrates new discoveries in physics, engineering, manufacturing and computer science into computational breakthrough approaches to solve most complex challenges.
			}
		\item IBM Quantum Computing
			\\ {\scriptsize Offers \href{https://quantumexperience.ng.bluemix.net/}{quantum experience} on cloud-enabled quantum computing platform
			\\ allows user to run algorithms, experiments, work on qubits, simulation etc.}
		\item Microsoft Quantum Computing
			\\ {\scriptsize Conducts theoretical and experimental approaches to creating quantum computers}
		\item Google Research
			\\ {\scriptsize
				Quantum Artificial Intelligence Lab - joint initiative of NASA, Universities Space Research Association and Google Research
				\\ goal - how quantum computing help with machine learning and other complex problems of computer science
				\\ Lab hosted at NASA's Ames Research Center
			}
		\item \href{www.toshiba.eu/eu/Cambridge-Research-Laboratory/Quantum-Information-Group/}{Toshiba Quantum Information Group}
			\\ {\scriptsize
				Research teams on Quantum Information, Speech Technology and Computer Vision.
				\\ Collaboration with Cavendish Laboratory and Engineering Department of the University, Cambridge and Toshiba R\&D groups.
			}
		\item \href{https://newsroom.intel.com/tag/qutech/}{Intel}
			\\ {\scriptsize 
				10 year collaboration with institute \href{https://qutech.nl/qutechentersintocollaborationwithintel/}{QuTech}, Netherlands formed in 2013 by Delft University of Technology for Applied Research in Quantum Computers
			}
		\item Alibaba Quantum Computing Laboratory
			\\ {\scriptsize
				goal - bring study and applications to the next level, platform for connectivity, computing and information security
			}
		\item Cambridge Quantum Computing
			\\ {\scriptsize
				independent company - expertise in Quantum Information Processing, AI, Optimization and pattern recognition
			}
		\item And many more...
			\\ {\scriptsize
					HP Lab : Quantum Information Processing, 1QB Information Technologies, Lockheed Martin, Regetti, IONQ, QxBranch, Post-Quantum, ID Quantique, QuintessenceLabs, Quantum Biosystems
			}
	\end{itemize}
\end{frame}

\begin{frame}
\frametitle{Types of Quantum Processor}
	\begin{block}{\textbf{Silicon Spin Qubits}}
		Electrons or nuclear spins on a solid subtract
	\end{block}
	\begin{block}{\textbf{Superconducting Circuits}}
		currents superposition around superconductor
	\end{block}
	\begin{block}{\textbf{Ion's Trap}}
		Trap ions in electric fields
	\end{block}
	\begin{block}{\textbf{Photonic Circuits}}
		qubits are photons driven in silicon circuits
	\end{block}

%	\begin{table}[H]
%	\centering
%		\begin{tabularx}{\linewidth}{ r X }
%			\textbf{Silicon Spin Qubits}
%			& 
%			Electrons or nuclear spins on a solid subtract
%			\\ 
%			\textbf{Superconducting Circuits}
%			& 
%			currents superposition around superconductor
%			\\ 
%			\textbf{Ion's Trap}
%			& 
%			Trap ions in electric fields
%			\\
%			\textbf{Photonic Circuits}
%			&
%			qubits are photons driven in silicon circuits
%		\end{tabularx}
%	\end{table}
\end{frame}


\end{document}

