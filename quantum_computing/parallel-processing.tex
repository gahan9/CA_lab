\documentclass[xcolor=x11names,table]{beamer}
\usepackage{beamerthemeCambridgeUS}  % CambridgeUS theme
\usepackage{float}  % perfect fit graphic with command [H]
%\usepackage{tabularx}  % alternate to tabular to use X to wrap column data properly
%\usetheme{Antibes}
\usepackage{hyperref}

%package for timeline
\usepackage[utf8]{inputenc}
\usepackage[english]{babel}
\usepackage[TS1,T1]{fontenc}
\usepackage{fourier, heuristica}
\usepackage{array, booktabs}
%\usepackage[x11names]{xcolor}
%\usepackage{caption}
%\DeclareCaptionFont{blue}{\color{LightSteelBlue3}}

\newcommand{\foo}{\color{LightSteelBlue3}\makebox[0pt]{\textbullet}\hskip-0.5pt\vrule width 1pt\hspace{\labelsep}}



% for themes, etc.
%\mode<presentation>
%{ \usetheme{boxes} }

\usepackage{times}  % fonts
\usepackage{graphicx, wrapfig, caption} % for graphics
\usepackage{multimedia}  % for media playback
\usepackage{multirow}



%colours
\definecolor{lightblue}{rgb}{0.1, 0.1, 0.6}
\definecolor{maroon}{rgb}{0.3, 0.1, 0.7}

\newcommand{\FR}[2]{
	{\textstyle \frac{#1}{#2} }}
\def\RR{\mathbb{R}}



% macros
\newcommand{\hhq}{{\scriptstyle{{\frac{1}{4}}}}}
\newcommand\hf{\textstyle{1\over 2 }\displaystyle}
\newcommand\hhf{\scriptstyle{1\over 2 }\displaystyle}
\newcommand{\erf}{\mathrm{erf}}
\def\h{\textcolor{red}{\mathbf{h}}}
\def\z{\textcolor{maroon}{\mathbf{z}}}
\newcommand{\zave}{\z_{\mathrm{ave}}}
\def\by{\textcolor{lightblue}{\mathbf{y}}}
\def\bv{\textcolor{blue}{\mathbf{v}}}
\def\bx{\textcolor{red}{\mathbf{x}}}
\def\bp{\textcolor{maroon}{\mathbf{p}}}
\makeatother
\setbeamertemplate{footline}
{
	\leavevmode%
	\hbox{%
		\begin{beamercolorbox}[wd=.4\paperwidth,ht=2.25ex,dp=1ex,center]{author in head/foot}%
			\usebeamerfont{author in head/foot}\insertshortauthor
		\end{beamercolorbox}%
		\begin{beamercolorbox}[wd=.6\paperwidth,ht=2.25ex,dp=1ex,center]{title in head/foot}%
			\usebeamerfont{title in head/foot}\insertshorttitle\hspace*{3em}
			%			\insertframenumber{} / \inserttotalframenumber\hspace*{1ex}
	\end{beamercolorbox}}%
	\vskip0pt%
}
\makeatletter
\setbeamertemplate{navigation symbols}{}




% these will be used later in the title page
\title{Multi-programming}
\subtitle{An overview on Parallel Processing with Silicon, Graphics and Quantum Chips}
\author{Gahan Saraiya}
\institute{Firmware Development Engineer 
	\\ Intel Technology India Pvt Ltd}
\date{{\scriptsize March 2023}}

% note: do NOT include a \maketitle line; also note that this title
% material goes BEFORE the \begin{document}
	
% Recurring Outline for every section, with highlighting
\AtBeginSection[]
{ \begin{frame}<beamer> 
		\frametitle{Outline of Talk}
		\tableofcontents[currentsection]%[pausesections]
\end{frame} }
	
\begin{document}
	
	\begin{frame}
		\titlepage
	\end{frame}
	
	
	\section{Introduction}
	
	
	\begin{frame}[allowframebreaks]
		\frametitle{Overview of Coverage}
		\begin{itemize}
			\item Basics of Parallel Processing
			{\scriptsize \\ Multi-programming, Multi-core Programming}
			\item Computing Core Types
				{\scriptsize 
					\\ CPU - Central Processing Unit
					\\ GPU - Graphics Processing Unit
					\\ QPU - Quantum Processing Unit
				}
			\item Complexity of Solution
				{\scriptsize
					\\ Complexity of implementation with CPU, GPU, QPU
				}
			\item Scalability of Architecture
		\end{itemize}
		%\vspace{1cm}
	\end{frame}
	
	\section{Basics of Parallel Processing}
	\begin{frame}
		\frametitle{Types of parallel processing}
		\begin{itemize}
			\item Single Instruction, Single Data (SISD)
			\item Multiple Instruction, Single Data (MISD)
			\item Single Instruction, Multiple Data (SIMD)
			\item Multiple Instruction, Multiple Data (MIMD)
			\item Single Program, Multiple Data (SPMD)
			\item Massively Parallel Processing (MPP)
		\end{itemize}
		%\vspace{1cm}
	\end{frame}
	
	\section{Compute Architectures}
	\begin{frame}
		\frametitle{CPU - Central Processing Unit}
		\begin{figure}[p]
%			\caption{Reference: https://www.researchgate.net/figure/High-level-architecture-of-modern-processors_fig7_323941692}
			\centering
			\includegraphics[width=\linewidth,height=\dimexpr\textheight-2\baselineskip-\abovecaptionskip-\belowcaptionskip\relax,keepaspectratio]{refs/High-level-architecture-of-modern-processors.png}
			\label{fig:cpu-architecture}
		\end{figure}
	
		%\vspace{1cm}
	\end{frame}
	
	\begin{frame}[allowframebreaks]
		\frametitle{GPU - Graphics Processing Unit}
		\begin{figure}[!h]
			\centering
			\includegraphics[keepaspectratio=true,height=0.8\textheight]{refs/memory-hierarchy.png}
			\label{fig:gpu-architecture}
		\end{figure}
		\framebreak
		
		\begin{table}
  			\newcommand{\ColWidth}{0.3\linewidth}
			% https://tex.stackexchange.com/a/9938
			\renewcommand\_{\textunderscore\allowbreak}
			\begin{tabular}{|p{2in}|p{2in}|} \hline
				CPU & GPU  \\ \hline \hline
				<100 (typically 4 to 8) cores & 100s or 1000s of cores \\ \hline
				Low latency & High Throughput \\ \hline
				Serial Processing & Parallel Processing \\ \hline
				Quick for Interactive Tasks & Breaks jobs into separate tasks to process simultaneously \\ \hline
				Traditional Program for sequential execution & Requires Additional software to convert CPU function to GPU functions 	
				\\ \hline
			\end{tabular}
		\end{table}
	
		\scriptsize {
			More details at -  \href{https://www.run.ai/guides/multi-gpu/cpu-vs-gpu}{\textbf{\textcolor{blue}{link 1}}} 
			,
			\href{https://www.weka.io/learn/hpc/cpu-vs-gpu}{\textbf{\textcolor{blue}{link 2}}}
			,
			\href{https://github.com/NVIDIA/cuda-samples/blob/master/Samples/0_Introduction/vectorAdd/vectorAdd.cu}{\textbf{\textcolor{blue}{Sample Code}}}
		}
		\framebreak
		\begin{figure}[p]
			\centering
			\includegraphics[width=\linewidth,height=0.8\textheight,keepaspectratio]{refs/gpu-devotes-more-transistors-to-data-processing.png}
			\caption{Block Diagram of Memory Architecture of CPU and GPU}
			\label{fig:cpu-gpu-architecture}
		\end{figure}
	\end{frame}
		
	\section{Quantum Architecture}
	
	\begin{frame}
		\centering
		Quantum Mechanics at the core of what we use on everyday basis -- \href{https://www.youtube.com/watch?v=hLtfFJ6F3rc}{\textbf{\beamergotobutton{Qiskit - Quantum Isn't Spooky.}}}
	\end{frame}

	\begin{frame}
		\frametitle{QPU - Quantum Processing Unit}
		\begin{figure}[p]
			\centering
			\includegraphics[width=\linewidth,height=0.8\textheight,keepaspectratio]{refs/quantum_layers.jpg}
			\caption{Layers of Quantum Computing}
			\label{fig:qpu-architecture-nvidia-qoda}
		\end{figure}
	\end{frame}


	\begin{frame}[allowframebreaks]
		\frametitle{Quantum Roadmap}
		\begin{figure}[p]
			\centering
			\includegraphics[width=\linewidth,height=0.7\textheight,keepaspectratio]{refs/ibm-quantum-development-roadmap.png}
			\caption{IBM Quantum Development Roadmap}
			\label{fig:ibm-quantum-roadmap}
		\end{figure}
	\end{frame}

	\begin{frame}[allowframebreaks]
		\frametitle{Applications Require Massively Parallel Computation}
		\begin{itemize}
			\item \textbf{Optimization}
			\begin{itemize}
				\item systems design
				\item airline scheduling
			\end{itemize}
			
			\item \textbf{Machine Learning}
			\begin{itemize}
				\item Improving forecast capability with neural network
				\item learn to recognize essences of objects by recognizing patterns in huge amount of data
			\end{itemize}
			
			\item \textbf{Biomedical Simulations}
			\begin{itemize}
				\item simulate molecular structures
			\end{itemize}
			
			\item \textbf{Financial Services}
			\begin{itemize}
				\item complex financial modeling and risk management within the financial industry
			\end{itemize}
		\end{itemize}
	\end{frame}
	
	\section{References}
	
	\begin{frame}
		\frametitle{References}
		\centering
		\href{https://www.youtube.com/watch?v=dx98pqJvZVk}{\beamergotobutton{Multicore Programming}} \\
		\href{https://www.youtube.com/watch?v=OWJCfOvochA}{\beamerbutton{Quantum Computing Expert Explains One Concept in 5 Levels of Difficulty | WIRED}} \\
		\href{https://www.youtube.com/watch?v=zhQItO6_WoI}{\beamergotobutton{Quantum Computers Explained in a Way Anyone Can Understand}} \\
		\href{https://www.explainingcomputers.com/quantum.html}{\beamergotobutton{Quantum Computing 2022 Update}} \\
%		\href{}{\beamergotobutton{}} \\
%		\href{}{\beamergotobutton{}} \\
%		\href{}{\beamergotobutton{}} \\
	\end{frame}

	
%	\begin{frame}[allowframebreaks]
%		\frametitle{Top Quantum Computing Companies}
%		\begin{itemize}
%			\item \href{https://www.dwavesys.com/}{D-wave Systems}
%			\\ {\scriptsize
%				World's first Quantum Computing Company
%				\\ integrates new discoveries in physics, engineering, manufacturing and computer science into computational breakthrough approaches to solve most complex challenges.
%			}
%			\item IBM Quantum Computing
%			\\ {\scriptsize Offers \href{https://quantumexperience.ng.bluemix.net/}{quantum experience} on cloud-enabled quantum computing platform
%				\\ allows user to run algorithms, experiments, work on qubits, simulation etc.}
%			\item Microsoft Quantum Computing
%			\\ {\scriptsize Conducts theoretical and experimental approaches to creating quantum computers}
%			\item Google Research
%			\\ {\scriptsize
%				Quantum Artificial Intelligence Lab - joint initiative of NASA, Universities Space Research Association and Google Research
%				\\ goal - how quantum computing help with machine learning and other complex problems of computer science
%				\\ Lab hosted at NASA's Ames Research Center
%			}
%			\item \href{www.toshiba.eu/eu/Cambridge-Research-Laboratory/Quantum-Information-Group/}{Toshiba Quantum Information Group}
%			\\ {\scriptsize
%				Research teams on Quantum Information, Speech Technology and Computer Vision.
%				\\ Collaboration with Cavendish Laboratory and Engineering Department of the University, Cambridge and Toshiba R\&D groups.
%			}
%			\item \href{https://newsroom.intel.com/tag/qutech/}{Intel}
%			\\ {\scriptsize 
%				10 year collaboration with institute \href{https://qutech.nl/qutechentersintocollaborationwithintel/}{QuTech}, Netherlands formed in 2013 by Delft University of Technology for Applied Research in Quantum Computers
%			}
%			\item Alibaba Quantum Computing Laboratory
%			\\ {\scriptsize
%				goal - bring study and applications to the next level, platform for connectivity, computing and information security
%			}
%			\item Cambridge Quantum Computing
%			\\ {\scriptsize
%				independent company - expertise in Quantum Information Processing, AI, Optimization and pattern recognition
%			}
%			\item And many more...
%			\\ {\scriptsize
%				HP Lab : Quantum Information Processing, 1QB Information Technologies, Lockheed Martin, Regetti, IONQ, QxBranch, Post-Quantum, ID Quantique, QuintessenceLabs, Quantum Biosystems
%			}
%		\end{itemize}
%	\end{frame}
	
%	\begin{frame}
%		\frametitle{Types of Quantum Processor}
%		\begin{block}{\textbf{Silicon Spin Qubits}}
%			Electrons or nuclear spins on a solid subtract
%		\end{block}
%		\begin{block}{\textbf{Superconducting Circuits}}
%			currents superposition around superconductor
%		\end{block}
%		\begin{block}{\textbf{Ion's Trap}}
%			Trap ions in electric fields
%		\end{block}
%		\begin{block}{\textbf{Photonic Circuits}}
%			qubits are photons driven in silicon circuits
%		\end{block}
%	\end{frame}
	
%	\begin{frame}
%		\frametitle{Timeline}
%		\begin{table}
%			\renewcommand\arraystretch{1.1}\arrayrulecolor{LightSteelBlue3}
%			%		\captionsetup{singlelinecheck=false, font=blue, labelfont=sc, labelsep=quad}
%			%		\caption{Timeline}\vskip -1.5ex
%			\begin{tabular}{@{\,}r <{\hskip 2pt} !{\foo} >{\raggedright\arraybackslash}p{5cm} l}
%				%			\toprule
%				%			\addlinespace[1.5ex]
%				May, 2011 & D-Wave One (Ranier) & 128qb \\
%				2013 & D-Wave Two & 512 qb \\
%				2015 & D-Wave 2X & 1152 qb \\
%				2016 & IBM Q Experience 5 & 5qb \\
%				2017 & Google & 20 qb \\
%				2017 & D-Wave 2000Q & 2000 qb \\
%				May, 2017 & IBM Q 16 & 16 qb \\
%				May, 2017 & IBM Q 17 & 17 qb \\
%				October, 2017 & Intel 17-Qubit Superconducting Test Chip & 17 qb \\
%				November, 2017 & IBM Q 20 & 20 qb \\
%				2017 & Rigetti 19Q & 19 qb \\
%				January 2018 & Intel Tangle Lake & 49 qb \\
%				March 2018 & Google Bristlecone & 72 qb \\
%			\end{tabular}
%		\end{table}
%	\end{frame}
	
	

	
%	\begin{frame}[allowframebreaks]
%		\frametitle{Google AI Quantum}
%		\begin{block}{Superconducting qubit processors}
%			Superconducting qubits with chip-based scalable architecture targeting two-qubit gate error < 0.5\%. Google's Bristlecone is newest 72-qubit quantum processor\footnote{as on 2018}.
%		\end{block}
%		\begin{block}{Qubit metrology}
%			Reducing two-qubit loss below 0.2\% is critical for error correction.
%			\\ Working on a quantum supremacy experiment, to approximately sample a quantum circuit beyond the capabilities of state-of-the-art classical computers and algorithms.
%		\end{block}
%		\begin{block}{Quantum simulation}
%			Simulation of physical systems is among the most anticipated applications of quantum computing.
%			\\ quantum algorithms for modelling systems of interacting electrons with applications in chemistry and materials science.
%		\end{block}
%		\begin{block}{Quantum neural networks}
%			a framework to implement a quantum neural network on near-term processors.
%			\\ understanding advantages may arise from generate massive superposition states during operation of the network.
%		\end{block}
%		\begin{block}{Quantum assisted optimization}
%			Developing hybrid quantum-classical solvers for approximate optimization.
%			\\ Thermal jumps in classical algorithms to overcome energy barriers could be enhanced by invoking quantum updates.
%			\\ Particular interested in coherent population transfer.
%		\end{block}
%	\end{frame}
	
	%	\begin{table}[H]
		%	\centering
		%		\begin{tabularx}{\linewidth}{ r X }
			%			\textbf{Silicon Spin Qubits}
			%			& 
			%			Electrons or nuclear spins on a solid subtract
			%			\\ 
			%			\textbf{Superconducting Circuits}
			%			& 
			%			currents superposition around superconductor
			%			\\ 
			%			\textbf{Ion's Trap}
			%			& 
			%			Trap ions in electric fields
			%			\\
			%			\textbf{Photonic Circuits}
			%			&
			%			qubits are photons driven in silicon circuits
			%		\end{tabularx}
		%	\end{table}
	
	
\end{document}
